% --------------------------------------
% Master's Thesis Title Page
% LaTeX Template
% Version 1.0 (23/05/14)
% Thanks to Magnus Marthinsen, this thesis and template is made available for Master studentes at HVL Joint SE program (01.03.2021)
%---------------------------------------

%----------------------------------------------------------------------------------------
%	PACKAGES AND OTHER DOCUMENT CONFIGURATIONS
%----------------------------------------------------------------------------------------
\documentclass[a4paper]{report}
\usepackage{graphicx} % Required for box manipulation
\usepackage{helvet}
\usepackage{subfig}
\usepackage[utf8]{inputenc}
%\usepackage{natbib}
\usepackage[USenglish]{babel}
\usepackage[useregional]{datetime2}
\usepackage{pgfgantt}
\usepackage{listings}
\usepackage{wrapfig}
\usepackage{setspace}
\usepackage{parskip} % Used to create spaces between paragraphs
\usepackage{dirtytalk} % quotes by talk
\usepackage[hidelinks]{hyperref}
% \usepackage[acronym, toc]{glossaries}  % Used to add a wordlist/glossaries
\usepackage{mathtools}
\usepackage{amsfonts}
\usepackage{color, colortbl}
\usepackage{booktabs}
\usepackage{float}
\usepackage{csquotes}

%BIB by Adrian
\usepackage[backend=biber,style=numeric, urldate=long]{biblatex}
% See the references.bib file. Most Bibtex bibliographies in Computer Science can be found from dblp.org 
\addbibresource{references.bib}



% Glossary/wordlist
% \makeglossaries
% \input{glossaries.tex}

\begin{document}

%
% COLORS USED THROUGH THE REPORT
%
\definecolor{kb_red}{RGB}{96,2,4}
\definecolor{light_gray}{RGB}{160,160,160}
\definecolor{med_gray}{RGB}{96,94,94}
\definecolor{black}{RGB}{0,0,0}
\definecolor{white}{RGB}{155,155,155}
\definecolor{light_green}{RGB}{208,240,192}
\definecolor{light_red}{RGB}{255,204,203}

% CODE STYLE
\definecolor{javared}{rgb}{0.6,0,0} % for strings
\definecolor{javagreen}{rgb}{0.25,0.5,0.35} % comments
\definecolor{javapurple}{rgb}{0.5,0,0.35} % keywords
\definecolor{javadocblue}{rgb}{0.25,0.35,0.75} % javadoc

\makeatletter
\lst@Key{matchrangestart}{f}{\lstKV@SetIf{#1}\lst@ifmatchrangestart}
\def\lst@SkipToFirst{%
    \lst@ifmatchrangestart\c@lstnumber=\numexpr-1+\lst@firstline\fi
    \ifnum \lst@lineno<\lst@firstline
        \def\lst@next{\lst@BeginDropInput\lst@Pmode
        \lst@Let{13}\lst@MSkipToFirst
        \lst@Let{10}\lst@MSkipToFirst}%
        \expandafter\lst@next
    \else
        \expandafter\lst@BOLGobble
    \fi}
\makeatother

\lstset{language=Java,
basicstyle=\footnotesize,
keywordstyle=\color{javapurple}\bfseries,
stringstyle=\color{javared},
commentstyle=\color{javagreen},
morecomment=[s][\color{javadocblue}]{/**}{*/},
numbers=left,
captionpos=b,
frame=single,
breakatwhitespace=false,
breaklines=true,
numberstyle=\tiny\color{black},
stepnumber=1,
numbersep=10pt,
tabsize=2,
showspaces=false,
showstringspaces=false,
matchrangestart=t}

%Title page
\input{title_page.tex}
\titlePage
\pagebreak

\section*{Abstract}
% \acrlong{MDSE} is a ...

\section*{Acknowledgements}
First and foremost, I would like to thank
\pagebreak
\tableofcontents
\listoffigures
\listoftables

% \printglossary[nonumberlist]
% \printglossary[type=\acronymtype, nonumberlist]


\chapter{Introduction}
% \acrfull{SE} is an engineering discipline that focuses on the development of high-quality software systems \cite{TsuiFrankF2011Eose}. ...

\section{Context and Approach}
When a hospital receives a injured patient that is considered for orthopedic surgery, a CT scan is
often performed. The data from the CT scan are then displayed as a set of 2D images or a 3D model on a computer. This helps the medical personnel plan the surgery by understanding what the fractured bone mass looks like.
If a surgeon has a good understanding of the anatomy related to the fracture,
the surgery has less risk of complications or could requires less resources.
\section{Problem Description}
A problem with visualising the model in 2D the limited understanding of what the bone actually looks like, because of the lack of scale and depth. This applies to both 2D images and 3D models rendered in 3D. 
A possible solution is visualising the model in Augmented Reality or Virtual Reality to give medical personnel a good feel for what the problem area actually looks like.
VR has many potential benefits, and it is possible that the surgery planning process can use some of these. As the users are already looking at 3d models in 2d screens, visualising in VR could improve the surgeons overview and improve the patients safety. The entire planning process could also be more effective, by removing or reducing the need for 3D printed models, especially in cases with limited time. Therefore the possible research questions are as follows:

How Can VR technology improve surgery planning by making the process safer or more effective?
How Can VR technology give some of the same benefits as 3D printing gives today at a lower cost?

\section{Methodology}

The project should include a VR prototype of a standard where it is
user friendly enough to test with non technical subjects and with functionality
that is comparable to the use cases of a printed model.
The prototype will be tested on medical personnel to investigate the impact on the users anatomical understanding, how it effects the surgery and the efficiency of the planning process. 
The application should be available for further development and/or study.

Firstly, a prototype Vr viewer will be created with the help of related
open-source frameworks/software and guidance from both orthopedic surgeons and
developers with experience from medical technology.

To answer the research question, it is necessary to measure the
performance of the final application. This thesis will use qualitative methods
by interviewing related personnel to investigate the performance including
anatomical understanding, cooperation, and the effectiveness of the planning
process.
This will also be put in context to the existing solutions, possibly
by doing a direct comparison by using a 3D model, printed model and VR viewer
on the same case.
\section{Contribution}
\section{Outline}

\chapter{Background}\label{Background}
In this chapter, we will present some of the knowledge that our research is built upon. This theory is important to know in order understand the following chapters.


\subsection{ Existing solutions}

\subsubsection{ 3D printing }

An alternative to digital representation is to print the 3d model to inspect it physically\cite{mishra_virtual_2019}. This has many advantages, such as the surgeon being able to physically hold the bodypart, measure the model, try out equipment and practice with it.
The biggest drawback to 3D printing is that the printing process can take more than 24 hours depending on the model, which in some cases is too long. Another drawback is not having any digital tools such as transparency, displaying cross sections or being able to alter the model in any way. Having a physical model in plastic also means it needs support structures, which can get in the way or create a inaccurate representation of the fracture.

A possible future use case for this project is taking a quick look at a model in VR, and then deciding if a printed model is necessary, potentially saving time and resources.

\subsubsection{2D viewer}
There exists a wide range of computer programs to inspect CT images as 3D models, using the computer to interact with the model.
The current solution used by Helse Vest is materialise\cite{materialise}. Using a 2D viewer is fast and simple, but lacks the depth and scale of VR.

\section{Virtual Reality}
Virtual Reality is the use of VR technology to sense the users state and actions and augments sensory feedback to immerse the user in a 3D virtual enviroment\cite{virtual reality applications}.
The Virtual enviroment is the computer generated objects that the user interacts with. The virtual enviroment will often mimic properties in the real world, such as shape, color or functionality.
The enviroment is typically percieved by the user through a Head-mounted Display (HMD), sound and to some extent haptic feedback (vibration). The virtual reality system then ‘tricks’ the senses by displaying computer-generated stimuli that replace stimuli from the real world. With more specialized hardware other stimuli such as temperatue, smells and more is possible\cite{feelreal}.
To allow for the virtual enviroment to seem real, it must respond to the users actions. Current commercial Virtual Reality headsets tracks the users head and hands, and allows for button inputs\cite{quest}. Modern HMD's use 6 Degrees of Freedom (DOF), meaning that the user is tracked in three dimensional position and rotation\cite{6dof}. This is used by the VR application to simulate walking, picking up objects and more.

\subsection{Proffesional usage of VR}
While becoming more popular on mainstream entertainment, VR has been used in professional enviroments for a long time\cite{needed}. VR and AR is often used in the medical field for training or education becuase the real situation would be unpractical or dangerous\cite{vrineducation}. 

\subsubsection{advantages and disadvangages}
A disadvantage with wearing a HMD is fatigue, both physical fatigue caused by the weight or eye fatigue and motion sickness\cite{motionSickness}.
Eye fatigue is caused by image imperfections in the HMD\cite{visualcomfort}, and motion sickness is caused by sensory conflict. 
In a study on motion sickness, 59 \% of the subjects experienced motion sickness after a mean of 14 minutes, the remaining subjects did not experience any illness\cite{motionSickness}. According to a study on Motion sickness factors, it can be mitigated in several ways.
Some effective measures are reducing Field of View (FOV), latency between motion and the HMD, flickering, moving content in the virtual enviroment, using several stimuli (audio, haptics)\cite{virtualrealitysickness}.

An advantage to VR is being able to simulate the something physically out of reach, such as a planet in space, or the inside of a patients knee. For physically impaired users, this advantage is even more relevant.
VR can also simulate things that would otherwise be dangerous, for example an untrained surgeon performing surgery alone.
Other advantages comes from the fact that VR are more immersive than other mediums. This makes the user able to better feel stress or fear and makes it feasable to prepare personell for stressful situations, such as police. 
The added immersion also makes teaching or training more motivating for the student.\cite{immersive VR}

\section{Augmented Reality}
Augmented Reality (AR) is a ugrade of VR where the real world is mixed with the virtual enviroment. 
AR is used in medical, industrial and military devices to show important information while the user opererates some other device in real life. Examples of this is a surgeon viewing medical information during a surgery or a pilot viewing a Heads Up Display while flying\cite{virtual reality intro}.


\section{orthopedic surgery} \label{Machine Learning}
\subsection{history of medical imaging}



\chapter{Design and Implementation} \label{Design and Implementation}

\section{Demonstration} \label{demonstration}

\section{Development method}

\section{Code structure} \label{CodeStructure}


\chapter{Use cases} \label{Use Cases}

\chapter{Analysis and Assessment}\label{Analysis and Assessment}

\chapter{Discussion} \label{Discussion}

\chapter{Related Work} \label{Related Work}

There currently exists several alternatives to viewing medical data in VR, here are some of the alternatives:

Medical Holodeck~\cite{medical_holodeck_medicalholodeck_nodate} is made for surgeons to plan surgeries and education.

An Augmented reality viewer called Dicom Director also exists, but is not yet approved for clinical use.\cite{dicomdirectorcom_surgeons_nodate}

Other similar solutions are Materialise\cite{materialise_medical_nodate} and
Ceevra\cite{ceevra_inc_using_2019}.

Most current solutions are closed source premium services targeted at enterprise/medical institutions. Many of them are not yet approved for clinical use, and current solutions are difficult to use for people not used to Virtual Reality \emph{cite needed}

Some free general purpose VR viewers for 3D models also exist, but I have not found any with features related to medical use, collaboration or planning.

A study in visualising Patient data with VR \cite{vertemati_virtual_2019} implemented a VR viewer for DICOM data and tried to measure anatomical understanding compared to 2D images. The study did not investigate the efficiency of the planning phase (loading the model into the software took 1 hour), and it did not do a comparison to 3D printing.

\chapter{Conclusion} \label{Conclusion}

\chapter{Further Work} \label{Further Work}

\appendix
\chapter{Source code} \label{SourceCode}

The source code for the VR application is available at this URL: \url{https://github.com/tobias2912/VR-DICOM-viewer}.



%\bibliographystyle{splncs04}
%\bibliography{references}
\printbibliography

\end{document}i
