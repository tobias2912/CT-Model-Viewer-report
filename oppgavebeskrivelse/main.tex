\documentclass[11pt]{article}

\usepackage[utf8]{inputenc}
\usepackage{csquotes}
\usepackage[british]{babel}
\usepackage[dvipsnames]{xcolor}
\usepackage{a4wide}
\usepackage{xspace}
\usepackage{amsmath}
\usepackage{graphicx}
\usepackage{algorithm}
\usepackage{algpseudocode}
\usepackage{tikz}
\usetikzlibrary{shapes.misc, positioning}
\usepackage{listings}
\usepackage{verbatim}
\usepackage{listings}
\usepackage{hyperref}
\usepackage{markdown}
\usepackage[dateabbrev=false]{biblatex}
\usepackage{placeins}


\markdownSetup{pipeTables,tableCaptions}
\definecolor{dkgreen}{rgb}{0,0.6,0}
\definecolor{gray}{rgb}{0.5,0.5,0.5}
\definecolor{mauve}{rgb}{0.58,0,0.82}

\addbibresource{report.bib}

\lstset{frame=tb,
  language=Java,
  aboveskip=3mm,
  belowskip=3mm,
  showstringspaces=false,
  columns=flexible,
  basicstyle={\small\ttfamily},
  numbers=left,
  numberstyle=\tiny\color{gray},
  keywordstyle=\color{blue},
  commentstyle=\color{green},
  stringstyle=\color{mauve},
  breaklines=true,
  breakatwhitespace=true,
  tabsize=3
}
\hypersetup{
    colorlinks=true,
    frenchlinks=false,
    bookmarksopen=true,
    breaklinks=true,
    linkcolor=black,
    urlcolor=black,
    citecolor=black,
    pdftitle=Project proposal,
    pdfauthor={Tobias Eilertsen}
}

\lstdefinelanguage{kotlin}{
  comment=[l]{//},
  commentstyle={\color{gray}\ttfamily},
  emph={delegate, filter, first, firstOrNull, forEach, lazy, map, mapNotNull, println, return@},
  emphstyle={\color{OrangeRed}},
  identifierstyle=\color{black},
  keywords={abstract, actual, as, as?, break, by, class, companion, continue, data, do, dynamic, else, enum, expect, false, final, for, fun, get, if, import, in, interface, internal, is, null, object, override, package, private, public, return, set, super, suspend, this, throw, true, try, typealias, val, var, vararg, when, where, while,lateinit},
  keywordstyle={\color{blue}\bfseries},
  morecomment=[s]{/*}{*/},
  morestring=[b]",
  morestring=[s]{"""*}{*"""},
  morekeywords={[2]{@Autowired, @RestController, @RequestMapping, @Bean, Array, Byte, Double, Float, Int, Integer, Iterable, Long, Runnable, Short, String, @Valid, @RequestBody,@PostMapping, @GetMapping, @DeleteMapping, @PutMapping}},
  keywordstyle={[2]\color{BurntOrange}\bfseries},
  morekeywords={[3]{accountService, account, window,id,accountRequest}},
  keywordstyle={[3]{\color{DarkOrchid}\bfseries}},
  morekeywords={[4]{AccountRequest, AccountResponse, AccountController,AccountService,AccountCreationRequest}},
  keywordstyle={[4]{\color{Periwinkle}\textbf}},
  sensitive=true,
  stringstyle={\color{ForestGreen}\ttfamily},
}
\begin{document}

\title{Project proposal}
\subtitle{Preoperative fracture surgery using virtual reality}

\author{Tobias Eilertsen}
\maketitle


\section*{supervisors}
\begin{itemize}
  \item Harald Soleim
  \item Atle Geitung
  \item Daniel Patel
\end{itemize}
\section*{Client}
\begin{itemize}
  \item Haukeland Universitetssykehus, medisinsk teknisk avdeling
\end{itemize}

\begin{abstract}
  Before and during a surgery, it is important for medical personel to have a good understanding of the anatomy of a patient.
  The goal of this project is to create a better way for medical personal to inspect 3D models, making surgeries easier and safer.
\end{abstract}

\newpage
\section{introduction}


\subsection{introduction}

When a hospital recieves an injured patient that needs surgery, a CT scan is performed. The CT images are displayed as a 3D model on a computer that helps the medical personel plan the surgery by visualising bone mass or other tissue. If a surgeon has a good understanding of a problem, it can be possible to perform a surgery that has less risk of complications or requires less resources.


The problem with visualising the model in 2D a limited understanding of what the Bone/tissue actually looks like because of the lack of scale and depth. A possible solution is visualising the model in Augumented Reality or Virtual Reality to give medical personel a good feel for what the problem actually looks like. 

\section{ background } 

\subsection{ existing solutions}

\subsubsection{existing VR viewers}

There exists some alternatives to viewing a CT scan in VR, such as Medical Holodeck \cite{holodeck} which is made for surgeons to plan surgeries and education. Current solutions are difficult to use for people not used to Virtual Reality

An Augumented reality viewer called Dicom Director also exists, but is not approved for clinical use. \cite{dicomdirector} All current solutions are also 


\subsubsection { 3D printing }

An alternative to digital representation is to print the 3d model to inspect it physically \cite{virtualplanningand3dprinting}. This has many advantages, such as the surgeon being able to physically hold the bodypart, measure the model, try out equipment and practice with it.
The biggest drawback to 3D printing is that the printing process can take more than 24 hours depending on the model, which in some cases is too long. Another drawback is not having any digital tools such as transparency, displaying cross sections or being able to alter the model in any way. Having a physical model in plastic also means it needs support structures, which can get in the way or create a inaccurate representation.

A possible usecase for this project is taking a quick look at a model in VR, and then deciding if a printed model is necessary, potentially saving time and resources.

\subsubsection {2D viewer}

There exists a wide range of computer programs to inspect CT images as 3D models, using the computer to interact with the model.
The current solution used by Helse Vest is materialise \cite{materialise}. Using a 2D viewer is fast and simple, but lacks the depth and scale of VR.

\section{Research Questions}

Can VR technology improve surgery planning by making the process safer or more effective?
can VR technology give some of the same benefits as 3D printing gives today?

VR has many potential benefits, and it is possible that the surgery planning process can use some of these. As the users are already looking at 3d models, visualising in VR could improve the surgeons overview and improve the patients safety. The entire planning process could also be more effective, by removing or reducing the need for 3D printed models.

\section{Expected Results}
To test the theory, this thesis will include developing a VR/AR application and testing the application on relevant cases.
The final application will be tested on medical personel to investigate the impact of the users understanding and the efficiency of the planning process. 

\section{Literature and References}
Mini invasiv behandling av brudd i hælbeinet ved hjelp av 3D printing:
Norwegian presentation showing how 3d Printing is used to improve surgery planning

Virtual preoperative planning and 3D printing are valuable for the management of complex orthopaedic trauma
Article describing how 3D printing in planning can reduce surgical morbidity

A Virtual Reality Environment to Visualize Three-Dimensional Patient-Specific Models by a Mobile Head-Mounted Display
Article testing the usability of viewing models in VR for surgeons

A collaborative virtual reality environment for liver surgery planning
Article using VR specifically for liver surgery planning.

\section{Research Method}
    Firstly, a prototypope Vr viewer will be created with the help of related open-source frameworks/software and guidance from medical personel.
    
    To answer the research question, it is necessary to somehow measure the performance of the final application. This thesis will use qualitative methods by interviewing related personel to investigate the performance including anatomical understanding, cooperation, and the effectiveness of the planning process. 

\section{Thesis Outline}
Introduction(7-9 pages)

short description of how vr can be used to plan surgery

Background(10-12 pages)

Description of related medical processes, surgery and surgery planning.
describe current state of 3D printing and Vr usage in medical field.

problem(2-3 pages)

describe what the application hopes to solve

method(3-4 pages)

how the application is created and how final result will be measured

Implementation(10-15 pages) 

how the final application is designed

Evaluation(10-12 pages)

results after final evaluation

Conclusion and Future Work(5 pages)

if project was a success and how VR in pre op planning can be improved

\section{Project Plan}



\section{Conclusions and Outlook}







improve the surirgical planning process to make operations safer, more effective

method: using vr to get an effect, pilot case study
better understanding, more effecctive

Ensure that your topic is relevant to your case???

Focus on a small, manageable problem

Establish your research question
what is relation to topic
"Preoperative fracture surgery using virtual reality"

Find 1-2 text/professional books that handle the topic

Find 3-10 research papers about the topic

Read and analyze + write about them

Ensure that the research question is relevant to your case


\section{Literature and References}


\printbibliography

\end{document}
