\documentclass[11pt]{scrartcl}

\usepackage[utf8]{inputenc}
\usepackage{csquotes}
\usepackage[british]{babel}
\usepackage[dvipsnames]{xcolor}
\usepackage{a4wide}
\usepackage{xspace}
\usepackage{amsmath}
\usepackage{graphicx}
\usepackage{algorithm}
\usepackage{algpseudocode}
\usepackage{tikz}
\usetikzlibrary{shapes.misc, positioning}
\usepackage{listings}
\usepackage{verbatim}
\usepackage{listings}
\usepackage{hyperref}
\usepackage{markdown}
\usepackage[dateabbrev=false]{biblatex}
\usepackage{placeins}
\graphicspath{ {./images/} }

\markdownSetup{pipeTables,tableCaptions}
\definecolor{dkgreen}{rgb}{0,0.6,0}
\definecolor{gray}{rgb}{0.5,0.5,0.5}
\definecolor{mauve}{rgb}{0.58,0,0.82}

\addbibresource{report.bib}

\lstset{frame=tb,
  language=Java,
  aboveskip=3mm,
  belowskip=3mm,
  showstringspaces=false,
  columns=flexible,
  basicstyle={\small\ttfamily},
  numbers=left,
  numberstyle=\tiny\color{gray},
  keywordstyle=\color{blue},
  commentstyle=\color{green},
  stringstyle=\color{mauve},
  breaklines=true,
  breakatwhitespace=true,
  tabsize=3
}
\hypersetup{
    colorlinks=true,
    frenchlinks=false,
    bookmarksopen=true,
    breaklinks=true,
    linkcolor=black,
    urlcolor=black,
    citecolor=black,
    pdftitle=Project proposal,
    pdfauthor={Tobias Eilertsen}
}

\lstdefinelanguage{kotlin}{
  comment=[l]{//},
  commentstyle={\color{gray}\ttfamily},
  emph={delegate, filter, first, firstOrNull, forEach, lazy, map, mapNotNull, println, return@},
  emphstyle={\color{OrangeRed}},
  identifierstyle=\color{black},
  keywords={abstract, actual, as, as?, break, by, class, companion, continue, data, do, dynamic, else, enum, expect, false, final, for, fun, get, if, import, in, interface, internal, is, null, object, override, package, private, public, return, set, super, suspend, this, throw, true, try, typealias, val, var, vararg, when, where, while,lateinit},
  keywordstyle={\color{blue}\bfseries},
  morecomment=[s]{/*}{*/},
  morestring=[b]",
  morestring=[s]{"""*}{*"""},
  morekeywords={[2]{@Autowired, @RestController, @RequestMapping, @Bean, Array, Byte, Double, Float, Int, Integer, Iterable, Long, Runnable, Short, String, @Valid, @RequestBody,@PostMapping, @GetMapping, @DeleteMapping, @PutMapping}},
  keywordstyle={[2]\color{BurntOrange}\bfseries},
  morekeywords={[3]{accountService, account, window,id,accountRequest}},
  keywordstyle={[3]{\color{DarkOrchid}\bfseries}},
  morekeywords={[4]{AccountRequest, AccountResponse, AccountController,AccountService,AccountCreationRequest}},
  keywordstyle={[4]{\color{Periwinkle}\textbf}},
  sensitive=true,
  stringstyle={\color{ForestGreen}\ttfamily},
}
\begin{document}

\title{Project proposal}
\subtitle{Preoperative fracture surgery using virtual reality}

\author{Tobias Eilertsen}
\maketitle


\section*{Supervisors}
\begin{itemize}
  \item Harald Soleim
  \item Atle Geitung
  \item Daniel Patel
\end{itemize}
\section*{Client}
\begin{itemize}
  \item Haukeland Universitetssykehus, medisinsk teknisk avdeling
\end{itemize}

\begin{abstract}
  Before and during a surgery, it is important for medical personel to have a good understanding of the anatomy of a patient.
  The goal of this project is to create a better way for medical personal to inspect 3D models, making surgeries easier and safer.
\end{abstract}

\newpage
\section{introduction}


\subsection{introduction}

When a hospital recieves an injured patient that needs surgery, a CT scan is performed. The CT images are displayed as a 3D model on a computer that helps the medical personel plan the surgery by visualising bone mass or other tissue. If a surgeon has a good understanding of a problem, it can be possible to perform a surgery that has less risk of complications or requires less resources.


The problem with visualising the model in 2D a limited understanding of what the Bone/tissue actually looks like because of the lack of scale and depth. A possible solution is visualising the model in Augumented Reality or Virtual Reality to give medical personel a good feel for what the problem actually looks like. 

\section{ background } 

\subsection{ existing solutions}

% \subsubsection{existing VR viewers}

There exists several alternatives to viewing a CT scan in VR, such as Medical Holodeck \cite{holodeck} which is made for surgeons to plan surgeries and education. Current solutions are difficult to use for people not used to Virtual Reality \emph{cite needed}

An Augumented reality viewer called Dicom Director also exists, but is not approved for clinical use. \cite{dicomdirector} Most current solutions are also closed source premium services.

A study in visualising Patient data with VR \cite{vertemati_virtual_2019} implemented a VR viewer for DICOM data and tried to measure anatomical uderstanding compared to 2D images. The study did not investigate the efficiency of the planning phase (loading the model into the software took 1 hour), and it did not do a comparison to 3D printing.

\subsubsection { 3D printing }

An alternative to digital representation is to print the 3d model to inspect it physically \cite{virtualplanningand3dprinting}. This has many advantages, such as the surgeon being able to physically hold the bodypart, measure the model, try out equipment and practice with it.
The biggest drawback to 3D printing is that the printing process can take more than 24 hours depending on the model, which in some cases is too long. Another drawback is not having any digital tools such as transparency, displaying cross sections or being able to alter the model in any way. Having a physical model in plastic also means it needs support structures, which can get in the way or create a inaccurate representation of the fracture.

A possible future usecase for this project is taking a quick look at a model in VR, and then deciding if a printed model is necessary, potentially saving time and resources.

\subsubsection {2D viewer}

There exists a wide range of computer programs to inspect CT images as 3D models, using the computer to interact with the model.
The current solution used by Helse Vest is materialise \cite{materialise}. Using a 2D viewer is fast and simple, but lacks the depth and scale of VR.


\section{Research Questions}
VR has many potential benefits, and it is possible that the surgery planning process can use some of these. As the users are already looking at 3d models, visualising in VR could improve the surgeons overview and improve the patients safety. The entire planning process could also be more effective, by removing or reducing the need for 3D printed models. Therefore the possible research questions are as follows:

Can VR technology improve surgery planning by making the process safer or more effective?
Can VR technology give some of the same benefits as 3D printing gives today at a lower cost?


\section{Expected Results}
To test the theory, this thesis will include developing a VR/AR application and doing qualitative testing on cases with relevant personel.
The final application will be tested on medical personel to investigate the impact on the users anatomical understanding, how it effects the surgery and the efficiency of the planning process. 

\section{Literature and References}
\begin{itemize}
\item Mini invasiv behandling av brudd i hælbeinet ved hjelp av 3D printing:
\begin{description}
\item Norwegian presentation showing how 3d Printing is used to improve surgery planning
\end{description}


\item Virtual preoperative planning and 3D printing are valuable for the management of complex orthopaedic trauma
\begin{description}
\item Article describing how 3D printing in planning can reduce surgical morbidity
\end{description}


\item A Virtual Reality Environment to Visualize Three-Dimensional Patient-Specific Models by a Mobile Head-Mounted Display
\begin{description}
\item Article testing the usability of viewing models in VR for surgeons
\end{description}


\item A collaborative virtual reality environment for liver surgery planning
\begin{description}
\item Article using VR specifically for liver surgery planning.
\end{description}
\end{itemize}

\section{Research Method}
Firstly, a prototypope Vr viewer will be created with the help of related open-source frameworks/software and guidance from medical personel.

To answer the research question, it is necessary to somehow measure the performance of the final application. This thesis will use qualitative methods by interviewing related personel to investigate the performance including anatomical understanding, cooperation, and the effectiveness of the planning process. 

\section{Thesis Outline}
\begin{itemize}

\item Introduction(7-9 pages)

\begin{description}
\item short description of how vr can be used to plan surgery
\end{description}

\item Background(10-12 pages)

\begin{description}
\item Description of related medical processes, surgery and surgery planning.
\item describe current state of 3D printing and Vr usage in medical field.
\end{description}

\item problem(2-3 pages)

\begin{description}
\item Describe what the application hopes to solve
\end{description}

\item method(3-4 pages)

\begin{description}
\item How the application is created and how final result will be measured
\end{description}

\item Implementation(10-15 pages) 

\begin{description}
\item how the final application is designed
\end{description}

\item Evaluation(10-12 pages)

\begin{description}
\item results after final evaluation
\end{description}

\item Conclusion and Future Work(5 pages)

\begin{description}
\item if project was a success and how VR in pre op planning can be improved
\end{description}

\end{itemize}

\section{Project Plan}



\section{Conclusions and Outlook}

\section{Literature and References}


\printbibliography

\end{document}
